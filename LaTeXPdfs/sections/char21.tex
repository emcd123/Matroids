\documentclass[../main.tex]{subfiles}
\begin{document}
 
\begin{defn}
A base is a maximally independent subset of $\mathcal{I}.$\\
\noindent As seen previously all maximally independent sets in a matroid have the same cardinality.
\end{defn}

\begin{thm}
 Let $\mathcal{B}$ be a set of subsets of a finite set E. Then $\mathcal{B}$ is the collection of bases of a matroid on E if and only if $\mathcal{B}$ satisfies the following conditions:\\
 (B1) $\mathcal{B}$ is non-empty.\\
 (B2) If $B_1$ and $B_2$ are members of $\mathcal{B}$ and $x \in B_1 \setminus B_2$, then there is an element $y$ of $B_1 \setminus B_2$ such that $(B_1 \setminus \{x\}) \cup \{y\} \in \mathcal{B}$.
 \end{thm}

\noindent \Proof \\
\noindent By (I1) $\emptyset$ is always independet, so $\mathcal{B}$ must always contain at least the $\emptyset$, (B1) holds.\\
\noindent Let $B_1 ,B_2 \in \mathcal{B}, B_1 \neq B_2.$ \\
\noindent $|B_1| = |B_2|$ so (I3) does not directly apply here.\\
\noindent Let $x \in B_1 \setminus B_2 \implies x \in B_1, x \notin B_2$\\
\noindent $|B_1| = |B_1 \setminus \{x\}| + 1 \implies B_1 \setminus \{x\} \in \mathcal{I}$ but not in $\mathcal{B}$\\
\noindent $|B_2| = |B_1 \setminus \{x\}| + 1$ so now we can use (I3)\\
\noindent Now $\exists y \in B_2 \setminus B_1$ such that $(B_1 \setminus \{x\}) \cup \{y\} \in \mathcal{I}$\\
$|(B_1 \setminus \{x\}) \cup \{y\}| = |B_1 \setminus \{x\}| + 1 = |B_1| = ... = |B_r|$ as all maximal elements of $\mathcal{I}$ have the same cardinality $\implies (B_1 \setminus \{x\}) \cup \{y\}$ is maximal in $\mathcal{I}$\\
$\implies (B_1 \setminus \{x\}) \cup \{y\} \in \mathcal{B}$, (B2) holds.

\vspace{4mm}

\noindent Conversely, suppose that $\mathcal{B}$ satisfies (B1) and (B2).\\
\noindent By B1 $\mathcal{B}$ is always non-empty which shows I1 holds.
\\
\noindent By definition, a base $B_1 \in \mathcal{B}$ is a maximally independent subset of E. Then for all $B_i \in \mathcal{B}$ the subsets $b_{i,k} \subseteq B_i$  are independent. Therefore all the $b_{i,k}$ are in $\mathcal{I}.$\\ \noindent $\implies$ (I2) holds. Showing a matroid can be generated through the bases.
\\
\noindent Assume that (I3) fails. That, for $I_1, I_2 \in \mathcal{I}$, where $|I_1| = |I_2|+1,$\\ \noindent there $\exists y \in I_1 \setminus I_2$ such that $I_2 \cup \{y\} \in \mathcal{I}.$\\
\noindent Let $B_1, B_2 \in \mathcal{B}, |B_1| = |B_2|.$\\
\noindent Let $x \in B_1 \setminus B_2$ then $B_1 \setminus \{x\} \subset B_1.$ \\ \noindent $\implies B_1 \setminus \{x\}$ is independent.\\
\noindent Then there exists a $y \in B_2 \setminus (B_1 \setminus \{x\})$ such that $(B_1 \setminus \{x\}) \cup \{y\} \in \mathcal{B}$ from (B2). And if $(B_1 \setminus \{x\}) \cup \{y\} \in \mathcal{B}$ it is also in $\mathcal{I}$. A contradiction.\\ \noindent $\implies$ (I3) holds and we have a matroid.

\qed
\end{document}