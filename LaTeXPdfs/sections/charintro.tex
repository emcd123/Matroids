\documentclass[../main.tex]{subfiles}
\begin{document}
One of the most interesting properties of matroids is their resulting cryptomorphisms.\footnote{In mathematics, two objects, especially systems of axioms or semantics for them, are called cryptomorphic if they are equivalent but not obviously equivalent.\cite{wiki}}
I previously mentioned that matroids are very flexible systems that can be adapted in a number of diverse ways. The cryptomorpshims are the key to this. Once we find some property that a matroid satisfies we can then extend the axioms to include a form of that property. This leads to multiple characterisations of matroids and is what causes matroids to naturally appear so often in seemingly unrelated fields of mathematics such as algorithms,graph theory,finite geometry etc.\\
In this section we explore some of the more advantageous characterisations of matroids used in this project, namely the circuits and bases.
\end{document}