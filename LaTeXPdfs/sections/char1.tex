\documentclass[../main.tex]{subfiles}
\begin{document}

\subsection{Circuit characterization of a matroid}
In the paper of Oxley \cite{ox_paper} it is said that "A set in a matroid that is not independent is called dependent.The hereditary property,(I2), means that a matroid is uniquely determined by its collection of maximal independent sets, which are called bases, or by its collection of minimal dependent sets, which are called circuits." \\
\indent The circuits of a graph are a rather intuitive concept, and so it is natural to define the cycle matroid $M(G)$ of a graph in terms of its circuits. These circuits are the "edge sets of cycles in a graph $G".$ All dependent sets in a matroid contain a circuit.\\
\indent A circuit of a matroid is defined as a minimally dependent set due to the fact that a circuit is a dependent set with the minimum number of edges to be dependent. i.e if you remove one edge we have an independent set.

\begin{defn} By using (I1)–(I3) , it is not difficult to show that the collection $\mathcal{C}$ of circuits of a matroid M has the following three properties:\\
(C1) The empty set is not in $\mathcal{C}$\\
(C2) No member of $\mathcal{C}$ is a proper subset of another member of $\mathcal{C}$\\
(C3) if $ C_1 $ and $ C_2 $ are distinct members of $ C $ and 
$ e \in C_1 \cap C_2 $, then $ (C_1 \cup C_2 ) \setminus \{e\} $ contains a member of $\mathcal{C}$ 
 \end{defn}
 
\noindent The following two proofs appear in Oxley's text and shows how the circuits define a matroid and how circuits appear in graphs.\cite{ox_book}

 \begin{thm}
 Let M be a matroid and $\mathcal{C}$ be its collection of circuits. Then $\mathcal{C}$ satisfies (C1) - (C3)
  \end{thm}
\begin{proof}
\noindent $(C1)$ is obvious as by $(I1)$ the empty set must always be an independent set.
 
\noindent $(C2)$ is also straightforward because any $ C \in \mathcal{C} $ is a minimally independent set by definition. Therefore , if there exists a $ C_1 \in \mathcal{C} $ such that $ C_1 \subset C $ then $ C_1 \in \mathcal{C} $ and $ C $ is not a minimally dependent subset of E. 
 
 \vspace{2mm}
 
 \noindent $(C3)$ Let $ A, B \in \mathcal{C} $ and suppose that (seeking a contradiction) $ (A \cup B) \setminus \{e\} $ where $ e $ is $ \in (A \cap B) $ does not contain a circuit.\\
 \noindent Then $ (A \cup B) \setminus \{e\} $ is independent and therefore in $\mathcal{I}.$
 
 \vspace{2mm}
 
\noindent The set $ A \setminus B $ is non-empty.\\
 Let $ s \in A \setminus B    \implies s \in A$\\
 \noindent as $A$ is in $\mathcal{C}$ it is minimally dependent.
 \noindent $\implies A \setminus \{s\} \in \mathcal{I}$ i.e is independent.
 
 \vspace{2mm}
 
\noindent Let $J$ be a maximal independent set of $(A \cup B)$ with the following properties: $A \setminus \{s\} \subset J $ and therefore $ \{s\} \notin J $ but as B is a circuit there must be some element $t \in B$ that is not in $J$; $s$ and $t$ are distinct.
 
 \vspace{2mm}
 
\noindent $\implies |J| $ must be at most equal to $|(A \cup B) \setminus \{s,t\}| $\\
 $\implies |J| \leq |(A \cup B) \setminus \{s,t\}| = |(A \cup B)| - 2 < |(A \cup B) \setminus \{e\}|$
 
 \vspace{2mm} 
 
 \noindent Now by (I3) we can substitute elements from $(A \cup B) \setminus \{e\}$ into $(A \cup B) \setminus \{s,t\}$ that are not in $(A \cup B) \setminus \{s,t\}$ but the only elements that fits this condition are $\{s\}$ and $\{t\}$ and introducing either of these elements breaks the independence of $J$.
 
\noindent Therefore, $(A \cup B) \setminus \{e\}$ must contain a circuit\\
 \end{proof}
 
\end{document}