
\documentclass{beamer}
 
\usepackage[utf8]{inputenc}
\usepackage{tikz,graphicx}
\newcommand\Algorithm{%
  \textbf{Algorithm:}\\%
}
\newcommand\Notation{%
  \textbf{Notation}\\%
}
 
\usetheme{Szeged}
\usecolortheme{seahorse}
 
\newenvironment<>{examplesecond}[1]{%
  \setbeamercolor{block title}{fg=white,bg=white!100!black}%
  \begin{block}#2{#1}}{\end{block}} 
 
%Information to be included in the title page:
\title[Matroids] %optional
{Matroids for solving Optimisation Problems}
 
\subtitle{Greedy algorithm as a solution}

\author{Owen McDonnell}
\institute{NUI Galway}
\date{2018}
 
\begin{document}
 
\frame{\titlepage}
 
\begin{frame}
\frametitle{The Problem}
\begin{minipage}{.1\textwidth}
\begin{tikzpicture}
    \tikzstyle{every node}=[draw,shape=circle];
    \node (A) at (1*60:2) {$B$};
    \node (B) at (2*60:2) {$A$};
    \node (D) at (4*60:2) {$C$};
    \node (E) at (5*60:2) {$D$};
    \draw (B) -- (D)
          (B) -- (A)
          (A) -- (E)
          (B) -- (E)
          (A) -- (D)
          (D) -- (E);
\end{tikzpicture}
\end{minipage}
\hspace{2.5cm} \begin{minipage}{.55\textwidth}
\begin{block}{The Setup}
\begin{itemize}
\item $G$ is a connected graph
\item Each vertex is a town.
\item \noindent Cost is assigned to each edge
\item Corresponds to providing a rail link between the towns.
\end{itemize}
\end{block}
\pause
\begin{block}{The Statement}
\begin{itemize}
\item \noindent Finding the minimum cost of a railway linking all $n$ towns.\\
\item \noindent Corresponds to finding a minimal weighted spanning tree of $G.$
\end{itemize}
\end{block}
\end{minipage}
\end{frame}

\begin{frame}
\frametitle{A Greedy Algorithm (Kruskal)}
Kruskal's algorithm is a greedy algorithm that finds a minimum spanning tree for a connected weighted Graph.

\vspace{4mm}

\begin{block}{Algorithm}
$1)$ Create a graph $F$ containing just the vertices of $G.$\\
\pause
$2)$ Create a set $S = E(G)$; the edge set of $G.$ \\
\pause
$3)$ While $S$ is non-empty and $F$ is not yet spanning\\
\pause
\hspace{5mm} $3(a)$ Remove an edge with minimum weight from S.\\
\hspace{5mm} $3(b)$ If the removed edge introduces no cycles to $F$\\
\hspace{5mm} then add the edge to $F$\\
\end{block}


\end{frame}

\begin{frame}
\frametitle{Why Greedy works?}

Let $B_G$ be a spanning tree created through the greedy algorithm.
\begin{lemma}
If $(E,\mathcal{I})$ is a matroid $M,$ then $B_G$ is a solution to the optimization problem.
\end{lemma}

\vspace{3mm}
\pause
\begin{block}{Question}
But what is a matroid?
\end{block}
\end{frame}


\begin{frame}
\frametitle{Independence Systems and Matroids}
\begin{definition}
An \textit{independence system} is a pair $(E,\mathcal{S}),$ where $E$ is a set and $\mathcal{S}$ is a collection of sets satisfying:\\
\vspace{2mm}
\pause
\noindent (I1) $\mathcal{S}$ is non-empty.\\
\noindent (I2) $\mathcal{S}$ is a  hereditary subset of the power set of $E$.\\
\noindent The elements of $\mathcal{S}$ are called the \textit{independent sets}.
\end{definition}
\pause
\begin{definition}
A matroid is a pair $(E,\mathcal{I})$ with finite ground set E and $\mathcal{I}$ being a collection of independent subsets of E satisfying the above conditions (I1), (I2) with the following extra condition:

\vspace{2mm}
\pause
 \noindent (I3): If $ A $ and $ B $ are two independent sets in $\mathcal{I}$ and $|A|=|B|+1$, then there exists $ x \in A \setminus B $ such that $ B \cup \{ x \} $ is in $\mathcal{I}$
\end{definition}

\end{frame}

\begin{frame}
\frametitle{Bases of a Matroid}
 
\begin{definition}
A base is a maximal independent subset of $\mathcal{I}.$\\
\noindent All the maximally independent sets have the same cardinality, this is the \textit{rank} of the matroid.
\end{definition}
\begin{definition}
  Let $\mathcal{B}$ be a set of subsets of a finite set E. Then $\mathcal{B}$ is the collection of bases of a matroid on E if and only if $\mathcal{B}$ satisfies the following conditions:\\
 (B1) $\mathcal{B}$ is non-empty.\\
 (B2) If $B_1$ and $B_2$ are members of $\mathcal{B}$ and $x \in B_1 \setminus B_2$, then there is an element $y$ of $B_2 \setminus B_1$ such that $(B_1 \setminus \{x\}) \cup \{y\} \in \mathcal{B}$.
 \end{definition}
\end{frame}

\begin{frame}
\frametitle{Spanning trees are Bases}
\begin{definition}
A spanning tree $T$ of an undirected graph $G$ is a subgraph that is a \textit{tree} which includes all of the vertices of $G.$
\end{definition}

\hspace{1cm} \begin{minipage}{.2\textwidth}
 \begin{tikzpicture}
    \tikzstyle{every node}=[draw,shape=circle];
    \node (A) at (1*70:2) {$B$};
    \node (B) at (2*70:1) {$A$};
    \node (D) at (4*70:1) {$C$};
    \node (E) at (5*70:2) {$D$};
    \draw (B) -- (D)[blue];
    \draw (B) -- (A)[blue];
    \draw (A) -- (E);
    \draw (B) -- (E)[blue];
    \draw (A) -- (D)
          (D) -- (E);
  \end{tikzpicture}  
 \end{minipage}
\hspace{2cm} \begin{minipage}{.2\textwidth}
 \begin{tikzpicture}
    \tikzstyle{every node}=[draw,shape=circle];
    \node (A) at (1*80:2) {$B$};
    \node (B) at (2*80:1) {$A$};
    \node (D) at (4*80:1) {$C$};
    \node (E) at (5*80:2) {$D$};
    \draw (B) -- (D)[red];
	\draw (B) -- (A)[red];
    \draw (A) -- (E)[red];
    \draw (B) -- (E)
          (A) -- (D)
          (D) -- (E);
\end{tikzpicture}  
 \end{minipage}


We see that $\mathcal{B}$ (the collection of maximal elements of $\mathcal{I}$) corresponds to the set of spanning trees of the graph.
\end{frame}

\begin{frame}
\frametitle{Weight Function}
The optimization problem associated with $(E,\mathcal{S)}$ is the following: for a given weight function $\omega : E \rightarrow \mathbb{R^{+}}$, we want to find an independent set $A$ whose weight is maximal, where
\begin{equation}
\omega(A) := \sum_{e \in A} \omega (e)
\end{equation}

\end{frame}

\begin{frame}
\frametitle{Demonstration of Kruskal's algorithm}
\begin{minipage}{.5\textwidth}
\includegraphics[scale=0.45]{kruskal1}
\end{minipage}
\hspace{5mm} \begin{minipage}{.4\textwidth}
\begin{tikzpicture}
    \tikzstyle{every node}=[draw,shape=circle];
    \node (A) at (1*60:3) {$B$};
    \node (B) at (2*60:2) {$A$};
    \node (D) at (4*60:2) {$C$};
    \node (E) at (5*65:2) {$D$};
    \draw (B) -- (E) node [midway, fill=white] {2};
	\draw (B) -- (A) node [midway, fill=white] {6};
    \draw (D) -- (B) node [midway, fill=white] {4};
\end{tikzpicture}

\end{minipage}
\end{frame}

\begin{frame}
\frametitle{Why Greedy works: The solution}
Let $B_G$ be a base of a matroid generated by the greedy algorithm.
\begin{lemma}
If $(E,\mathcal{I})$ is a matroid $M,$ then $B_G$ is a solution to the optimization problem.
\end{lemma}
\pause
\begin{proof}
If $r(M) = r,$ then $B_G = \{e_1,e_2, ..., e_r\}$ is a basis of $M.$ \pause Let $B$ be another basis of $M$, $B = \{f_1, f_2, ..., f_r\}$ \pause
where $\omega(f_1) \geq \omega(f_2) \geq ... \geq \omega(f_r).$\pause We claim that $\omega(e_j) \geq \omega(f_f)$ $ \forall j$ , then it follows that $\omega(B_G) \geq \omega(B)$ for any other basis in $\mathcal{B}.$
\end{proof}
\end{frame}

\begin{frame}
\frametitle{Continued}
\begin{lemma}
If $1 \leq j \leq r,$ then $\omega(e_j) \geq \omega(f_j).$
\end{lemma}
\begin{proof}
Suppose (seeking a contradiction) that $k$ is the least integer for which $\omega(e_k) < \omega(f_k).$\pause Take $I_1 = \{e_1, e_2, ..., e_{k-1}\}$ and $I_2 = \{f_1, f_2, ..., f_{k}\}.$\pause Since $|I_2| = |I_1|+1$  $(I3)$ implies $I_1 \cup \{f_t\} \in \mathcal{I}$ for some $f_t \in I_2 \setminus I_1.$ \pause But this means that $\omega(f_t) \geq \omega(f_k) > \omega(e_k)$\\ \pause And hence the Greedy algorithm would have chosen $f_t$ over $e_k$, which gives us our contradiction.
\end{proof}

\end{frame}

\begin{frame}
\frametitle{Wrap Up}
\begin{itemize}
\item The previous proof shows that the greedy algorithm finds a maximal member of $B$ of $\mathcal{I}$ of maximum weight\\
\item This allows us to deduce that Kruskals algorithm is correct.\\
\pause
\item And  that the greedy algorithm gives us a solution to our optimisation problem as long as we have a matroid\\
\pause
\item Furthermore, the greedy does not provide a solution if the data does not form a matroid.
\end{itemize}
\end{frame}



\end{document}
