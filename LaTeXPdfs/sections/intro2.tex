\documentclass[../main.tex]{subfiles}
\begin{document}

\subsection{Enumeration of Matroids}
\begin{defn}
Let $\mathcal{I}$ be the collection of subsets of $E$ that do not contain all of the edges of any \textit{cycle} of $G.$
We get a matroid on the edge set of every graph $G$ by defining $\mathcal{I}$ in this way. This matroid is called the \textit{cycle matroid} of the graph $G$ and is denoted $M(G).$
\end{defn}

\begin{defn}
If $M_i, M_j$ are matroids , then there exists a bijection from the ground set of $M_i$ to the ground set of $M_j$, such that a set is independent in the first matroid if and only if it is independent in the second matroid, then $M_i$ and $M_j$ are said to be isomorphic. 
\end{defn}

\begin{note}
A matroid that is isomorphic to the cycle matroid of some graph is called graphic.
And every graphic matroid is binary
\end{note}

\noindent The numbers of non-isomorphic matroids and binary matroids on an n-element set for $0 \leq n \leq  8$
\begin{center}
 \begin{tabular}{| c c c c c c c c c c |} 
 \hline
 n & 0 & 1 & 2 & 3 & 4 & 5 & 6 & 7 & 8 \\ [0.5ex] 
 \hline\hline
 matroids & 1 & 2 & 4 & 8 & 17 & 38 & 98 & 306 & 1724\\ 
 \hline
 binary matroids & 1 & 2 & 4 & 8 & 16 & 32 & 68 & 148 & 342\\
 \hline
\end{tabular}
\end{center}
It can be seen from this table, that number of possible matroids on an $n$-set grows very rapidly.
\end{document}