\documentclass[../main.tex]{subfiles}
\begin{document}
The following algorithm is useful towards implementing the greedy algorithm as described in \textit{section 4} due to the fact that the element of the edge set of a graph $G$ are sorted in terms of their weights. The following process is a straightforward method to arrange your elements in the desired order.

\begin{algorithm}[H]
\caption{Insertion Sort}\label{sorting}
Let $A$ be an unsorted list containing real number positive values (to match our weight functions).
\begin{algorithmic}[1]
\Procedure{INSERTION}{$A$}
\State $i \gets 1$
\While{$i <$ length$(A)$}
\State $j \gets i$
\While{$j>0$ \textbf{and} $A[j-1] > A[j]$}
	\State  \textbf{Swap} $A[j]$ \textbf{and} $A[j-1]$
	\State $j \gets j-1$	
	\EndWhile
	\State $i \gets i+1$
\EndWhile
\EndProcedure
\end{algorithmic}
\end{algorithm}
This is a Perl5 implementation of an insertion sort I wrote, for small lists and graphs with a small number of edges this is a suitable sorting algorithms. For larger graphs, a more sophisticated algorithm such as quicksort might be more appropriate.
\begin{lstlisting}
use strict;
use warnings;
use Data::Alias;
#my @li = (31, 46, 10, 6, 23, 9, 29, 19, 3, 46);
sub InsertionSort{
	my $i = 1;
	while($i < scalar @li){
		my $j = $i;
		while($j>0 && $li[$j-1] > $li[$j]){
			alias @li[$j,$j-1] = @li[$j-1,$j];
			#$li[$j], $li[$j-1] = $li[$j-1], $li[$j];
			$j = $j-1;
		}
		$i = $i+1;
	}
}
InsertionSort(@li);
\end{lstlisting}


\end{document}