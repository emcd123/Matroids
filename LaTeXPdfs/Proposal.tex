\documentclass{article}
\usepackage[utf8]{inputenc}

\usepackage{amsmath,amsfonts,amssymb,amsthm,epsfig,epstopdf,titling,url,array, tikz,tkz-berge}
\usepackage{tkz-graph}

\title{Undergraduate Project Proposal}
\author{Owen McDonnell, 14361511}
\date{January 2018}


\theoremstyle{plain}
\newtheorem{thm}{Theorem}[section]
\newtheorem{lem}[thm]{Lemma}
\newtheorem{prop}[thm]{Proposition}
\newtheorem*{cor}{Corollary}

\theoremstyle{definition}
\newtheorem{defn}{Definition}[section]
\newtheorem{conj}{Conjecture}[section]
\newtheorem{exmp}{Example}[section]

\theoremstyle{remark}
\newtheorem*{rem}{Remark}
\newtheorem*{note}{Note}

\newenvironment{rcases}
  {\left.\begin{aligned}}
  {\end{aligned}\right\rbrace}

\newcounter{excercise}
\newcounter{solution}
\newcounter{Question}

\newcommand\Excercise{%
  \textbf{Excercise:}~%
  \setcounter{solution}{0}%
}

\newcommand\TheSolution{%
  \textbf{Solution:}\\%
}

\newcommand\Question{%
    \textbf{Question:}~%
    \setcounter{Question}{0}%
}
\newcommand\Notation{%
  \textbf{Notation:}~%
}

\newcommand\Proof{%
    \textbf{Proof:}~%
}

\setlength{\droptitle}{-10em}

\begin{document}
\maketitle

We plan to study matroids from scratch, following Oxley's text \cite{ox_book} together with \cite{ox_paper}. A matroid is a structure that abstracts and generalizes the notion of linear independence in vector spaces. There are many equivalent ways to define a matroid, the most significant being in terms of independent sets, bases, circuits, closed sets or flats, closure operators, and rank functions.\\
\begin{defn}
A matroid is a pair (E,\mathcal{I}) with finite ground set E and \mathcal{I} being a collection of independent subsets of E satisfying the following conditions

\vspace{2mm}

\noindent (I1): The empty set is always independent\\
\noindent (I2): Every subset of an independent set is independent\\
\noindent (I3): If $ A $ and $ B $ are two independent sets (i.e., each set is independent) of \mathcal{I} and $ A $ has more elements than $ B $, then there exists $ x \in A \setminus B $ such that $ B \cap \{ x \} $ is in \mathcal{I}
\end{defn}

\vspace{2mm}

We will consider the motivating problems and examples from Oxley's work and using a mathematical software package attempt to reproduce some currently best-known results of the enumeration problems. Our primary target is to explain the connection between matroids and the greedy algorithm.

\vspace{1mm}

The Greedy algorithm is an interesting characterization of matroids in that it arises frequently in problems related to combinatorial optimzation. For example the well known graph optimization problem: finding the minimum spanning tree. Leading to a connection between Kruskal's algorithm and matroids.

\begin{thebibliography}{}
\bibitem{ox_book}
    Oxley, James (1992), Matroid Theory, Oxford: Oxford University Press, ISBN 0-19-853563-5, MR 1207587, Zbl 0784.05002.
\bibitem{ox_paper}
    James Oxley : What is a matroid? https://www.math.lsu.edu/~oxley/survey4.pdf
    
    

\end{thebibliography}
\end{document}