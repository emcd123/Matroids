\documentclass{article}
\usepackage[utf8]{inputenc}

\usepackage{amsmath,amsfonts,amssymb,amsthm,epsfig,epstopdf,titling,url,array, tikz,tkz-berge, calrsfs}
\usepackage{tkz-graph}

\title{Matroids And their Graphs}
\author{o.mcdonnell4@nuigalway.ie }
\date{19 January 2018}


\theoremstyle{plain}
\newtheorem{thm}{Theorem}[section]
\newtheorem{lem}[thm]{Lemma}
\newtheorem{prop}[thm]{Proposition}
\newtheorem*{cor}{Corollary}

\theoremstyle{definition}
\newtheorem{defn}{Definition}[section]
\newtheorem{conj}{Conjecture}[section]
\newtheorem{exmp}{Example}[section]

\theoremstyle{remark}
\newtheorem*{rem}{Remark}
\newtheorem*{note}{Note}

\newenvironment{rcases}
  {\left.\begin{aligned}}
  {\end{aligned}\right\rbrace}

\newcounter{excercise}
\newcounter{solution}
\newcounter{Question}

\newcommand\Excercise{%
  \textbf{Excercise:}~%
  \setcounter{solution}{0}%
}

\newcommand\TheSolution{%
  \textbf{Solution:}\\%
}

\newcommand\Question{%
    \textbf{Question:}~%
    \setcounter{Question}{0}%
}
\newcommand\Notation{%
  \textbf{Notation:}~%
}

\newcommand\Proof{%
    \textbf{Proof:}~%
}

\setlength{\droptitle}{-10em}

\begin{document}
\maketitle
 
 \section{Matroids imply Circuits: Alternative C3}
 \begin{defn} By using (I1)–(I3) , it is not difficult to show that the collection $\mathcal{C}$ of circuits of a matroid M has the following three properties:\\
(C1) The empty set is not in $\mathcal{C}$\\
(C2) No member of $\mathcal{C}$ is a proper subset of another member of $\mathcal{C}$\\
(C3) if $ C_1 $ and $ C_2 $ are distinct members of $ C $ and 
$ e \in C_1 \cap C_2 $, then $ (C_1 \cup C_2 ) \setminus \{e\} $ contains a member of $\mathcal{C}$ .
 \end{defn}
 
 \vspace{5mm}
 
 \begin{thm}
 A matroid M $\implies$ all circuits in $\mathcal{C}$ satisfy (C1-C3).
 \end{thm}
 
 \noindent\Proof Let $A,b \in \mathcal{C}$, $A \neq B$
 \\
 \noindent Let $e \in A \cap B$
 \\
 \noindent Suppose(seeking a contradiction) that $(A \cap B) \setminus \{e\}$ is independent.
 \\
 \noindent Let $x \in A \setminus B$, noting that $x \neq e$
 \\
 \noindent $\implies A \setminus \{x\} \in \mathcal{I}$ i.e independent.
 \\
 \noindent We can add elements of $(A \cup B) \setminus \{e\}$ to $A \setminus \{x\}$ retaining independence until we have $|A \cup B| - 1$ elements.
 \\
 \noindent But now our new set $(A \cup B) \setminus \{x\}$ contains all of $B$ and $B \in \mathcal{C}$. Therefore, we have a contradiction, $(A \cup B) \setminus \{e\}$ cannot be independent
 \\
 \qed
 \end{document}