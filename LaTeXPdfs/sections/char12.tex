\documentclass[../main.tex]{subfiles}
\begin{document}
 %\subsection{Circuits in a graph}
 \begin{thm}
 Let $E$ be the edge set of a graph $G$ and let $\mathcal{C}$ be the edge sets of cycles in $G$.\\
 \noindent Then $\mathcal{C}$ is the set of circuits of a matroid.
 \end{thm}

\begin{proof}
Let $A, B \in \mathcal{C}, A \neq B $ and let $e \in A \cap B .$

\vspace{1mm}

\noindent We must now construct a minimal cycle of $G$ whose edge set is contained in $(A \cup B) \setminus \{e\}.$\\
\noindent For $ i = 1,2,3,...... $ let $P_1$ be a path whose edge set is $ A \setminus \{e\}.$\\
\noindent $ A \setminus \{e\} \in \mathcal{I} $ therefore $P_1$ is not a cycle of $G$. This path will traverse from the edge $a_j$ to $a_u$ where $u,j$ were the vertices connecting the edge e to $(A \cup B) \setminus \{e\}$ to make $A \cup B$.\\
\noindent Now perform the same procedure for a path $P_2$ whose edge set is $B \setminus \{e\}.$\\
\noindent $P_1$ and $P_2$ should meet at the junctions $u,v$,  where $e$ was removed to make $ (A \cup B) \setminus \{e\}.$\\
\noindent Therefore $P_1 \cup P_2$ should be a cycle of $G$.\\
\noindent $\implies$ (C3) holds.\\
\noindent $\implies \mathcal{C}$ is the set of circuits in $G$.\\
\end{proof}
\end{document}