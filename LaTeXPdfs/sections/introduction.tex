\documentclass[../main.tex]{subfiles}

\begin{document}

\subsection{Matroid}

\begin{defn}
An \textit{independence system} is a pair $(E,\mathcal{S}),$ where $E$ is a set and $\mathcal{S}$ is a collection of sets satisfying:\\
\noindent (I1) $\mathcal{S}$ is non-empty.\\
\noindent (I2) $\mathcal{S}$ is a  hereditary subset of the power set of $E$.\\
\noindent The elements of $\mathcal{S}$ are called the \textit{independent sets}.
\end{defn}

\begin{defn}
A matroid is a pair $(E,\mathcal{I})$ with finite ground set E and $\mathcal{I}$ being a collection of independent subsets of E satisfying the following conditions:

\vspace{2mm}

\noindent (I1): The empty set is always independent\\
\noindent (I2): Every subset of an independent set is independent\\
\noindent (I3): If $ A $ and $ B $ are two independent sets of $\mathcal{I}$ and $|A|>|B|$, then there exists $x \in A \setminus B$ such that $B \cup \{ x \}$ is in $\mathcal{I}$
\end{defn}

\begin{lem}
Prove that $(E,\mathcal{I})$ is a matroid if and only if $\mathcal{I}$ satisfies $(I2)$ and the following two conditions:\\
\noindent $(I1)' \mathcal{I} \neq \emptyset$\\
\noindent $(I3)'$ If $I_1,I_2$ are in $\mathcal{I}$ and $|I_2|=|I_1|+1$, then there is an element $e \in I_2 \setminus I_1$ such that $I_1 \cup \{e\} \in \mathcal{I}$
\end{lem}
\begin{proof}
Suppose $(E,\mathcal{I})$ is a matroid. Then by hypothesis, $\mathcal{I}$ satisies $(I2).$\\
\noindent By $(I1)$ the empty set is laways contained in $\mathcal{I}$, so $\mathcal{I}$ is always non-empty.\\
\noindent $\implies (I1)'$ holds.\\
\noindent Let $I_1,I_2 \in \mathcal{I}$ and $|I_2|>|I_1|$ then by $(I3)$ there exists $e \in I_2 \setminus I_1$ such that $I_1 \cup \{e\} \in \mathcal{I}$ by $(I3).$\\
\noindent But by $(I2)$ there is an $I_2'$ such that $e \in I_2'$ and $|I_2'|=|I_1|+1$ as a matorid is hereditary and $I_2' \subset I_2.$
$\implies \exists e \in I_2' \setminus I_1$ such that $I_1 \cup \{e\} \in \mathcal{I}.$\\
Conversely, Suppose $\mathcal{I}$ satisfies $(I2),(I1)',(I3)'$\\
\noindent By (I1)', $\mathcal{I}$ is always non-empty. (I2) and (I3) hold by hypothesis.
\end{proof}

\begin{exmp}
Let $M_1, M_2$ be matroids on a set $E.$ Let $E=\{1,2,3,4\}$\\
\noindent Let $\mathcal{I}_1 = \{\emptyset, \{1\},\{2\},\{3\},\{4\},\{1,2\},\{1,3\},\{2,4\},\{3,4\}\}$\\
\noindent Let $\mathcal{I}_2 = \{\emptyset, \{1\},\{2\},\{3\},\{4\},\{1,2\},\{1,4\},\{2,3\},\{3,4\}\}$\\
let $(E, \mathcal{I}_1 \cap \mathcal{I}_2)$ be a pair, is it a matroid?\\
Let $I_1 = \{1,2\}$ and $I_2=\{3\}$\\
If $\exists e \in I_1$ such that $I_2 \cup \{e\} \in \mathcal{I}$ then we have a matroid.\\
$I_2 \cup \{1\} = \{1,3\} \notin \mathcal{I}$,\\
$I_2 \cup \{2\} = \{2,3\} \notin \mathcal{I}$\\
\noindent $\implies (E, \mathcal{I}_1 \cap \mathcal{I}_2)$ is not a matroid.
\end{exmp}
\end{document}