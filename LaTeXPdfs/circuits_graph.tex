\documentclass{article}
\usepackage[utf8]{inputenc}

\usepackage{amsmath,amsfonts,amssymb,amsthm,epsfig,epstopdf,titling,url,array, tikz,tkz-berge, calrsfs}
\usepackage{tkz-graph}

\title{Matroids And their Graphs}
\author{o.mcdonnell4@nuigalway.ie }
\date{19 January 2018}


\theoremstyle{plain}
\newtheorem{thm}{Theorem}[section]
\newtheorem{lem}[thm]{Lemma}
\newtheorem{prop}[thm]{Proposition}
\newtheorem*{cor}{Corollary}

\theoremstyle{definition}
\newtheorem{defn}{Definition}[section]
\newtheorem{conj}{Conjecture}[section]
\newtheorem{exmp}{Example}[section]

\theoremstyle{remark}
\newtheorem*{rem}{Remark}
\newtheorem*{note}{Note}

\newenvironment{rcases}
  {\left.\begin{aligned}}
  {\end{aligned}\right\rbrace}

\newcounter{excercise}
\newcounter{solution}
\newcounter{Question}

\newcommand\Excercise{%
  \textbf{Excercise:}~%
  \setcounter{solution}{0}%
}

\newcommand\TheSolution{%
  \textbf{Solution:}\\%
}

\newcommand\Question{%
    \textbf{Question:}~%
    \setcounter{Question}{0}%
}
\newcommand\Notation{%
  \textbf{Notation:}~%
}

\newcommand\Proof{%
    \textbf{Proof:}~%
}

\setlength{\droptitle}{-10em}

\begin{document}
\maketitle
 \section{Circuits in a graph}
 
 \begin{thm}
 Let $E$ be the edge sets of a graph $G$ and let $\mathcal{C}$ be the edge sets of cycles in $G$.\\
 \noindent Then $\mathcal{C}$ is the set of circuits of a matroid.

 \end{thm}

\noindent\textbf\Proof
Let $A, B \in \mathcal{C}, A \neq B $ and let $e \in A \cap B $

\vspace{1mm}

\noindent We must now construct a cycle of $G$ whose edge set is contained in $(A \cup B) \setminus \{e\}$

\vspace{2mm}

\noindent For $ i = 1,2,3,...... $ let $P_1$ be a path whose edge set is $ A \setminus \{e\}$\\
\noindent $ A \setminus \{e\} \in \mathcal{I} $ therefore $P_i$ is not a cycle of $G$. This path will traverse from the edge $a_j$ to $a_u$ where $u,j$ were the vertices connecting the edge e to $(A \cup B) \setminus \{e\}$ to make $A \cup B$.

\vspace{2mm}

\noindent Now perform the same procedure for a path $P_2$ whose edge set is $B \setminus \{e\}$.

\vspace{2mm}

\noindent $P_1$ and $P_2$ should meet at the junctions $u,v$,  where $e$ was removed to make $ (A \cup B) \setminus \{e\}$\\
\noindent Therefore $P_1 \cup P_2$ should be a cycle of $G$.\\
\noindent $\implies$ (C3) holds\\
\noindent $\implies \mathcal{C}$ is the edge sets of cycle in $G$.\\
\qed
\end{document}