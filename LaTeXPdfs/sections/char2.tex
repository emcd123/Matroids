\documentclass[../main.tex]{subfiles}
\begin{document}
 \subsection{Base Characterisation of a Matroid}
As quoted above, the bases of a matroid are the maximally independent sets of $\mathcal{I}.$  These are most naturally visualised as being akin to a basis in a matrix. Where the basis of a matrix is a spanning set and every vector is a linearly combination of the basis. Bases are also spanning sets in  matroid. And as such have very useful properties that we will use extensively in later sections. Most notably, the collection of subsets $\mathcal{I}$ can be generated using the bases (bases are not in general unique) due to the hereditary property $(I2).$ \\
The below theorem gives us a useful property that can be applied to the bases of a matroid, namely that the bases all have the same cardinality.
 \begin{thm}
 Show that if $\mathcal{I}$ is a non-empty hereditary set of subsets of a finite set E, then $(E,\mathcal{I})$ is a matroid if and only if, for all $X \subset E$, all maximal members of $\{I : I \in \mathcal{I} $ and $ I \subset X\}$ have the same number of elements.
 \end{thm}
\begin{proof}
 Let $B_1 , B_2$ be maximal elements of $\{I : I \in \mathcal{I} $and $ I \subset X\}$ \\
\noindent Assume $|B_1| < |B_2|, B_1, B_2 \in \mathcal{I}$ and since we have a matroid
\\
there exists $e \in (B_2 \setminus B_1)$ such that $B_1 \cup \{e\} \in \mathcal{I}.$
\\
This contradicts the maximality of $B_1$.\\
 \noindent $\implies$ All maximal elements of the set $\{I : I \in \mathcal{I} $ and $ I \subset X\}$ in our matroid M have the same cardinality.

\end{proof}
 \end{document}