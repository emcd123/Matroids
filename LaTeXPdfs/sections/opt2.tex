\documentclass[../main.tex]{subfiles}
\begin{document}
\subsection{Cryptomorphism: Greedy}
The following theorem appears in Oxley's text\cite{ox_book} which gives us another cryptomorphism of matroids. From this proof we can see that the greedy algorithm is not optimal for hereditary systems unless that system is also a matroid.
\begin{thm}
Let $\mathcal{I}$ be a collection of subsets of a set $E.$ Then $(E,\mathcal{I})$ is a matroid if and only if $\mathcal{I}$ satisfies the following conditions:\\
$(I1)$ $\emptyset \in \mathcal{I}$\\
$(I2)$ If $I \in \mathcal{I}$ and $I' \subset I$ then $I' \in \mathcal{I}$\\
$(G)$ For all weight functions $\omega:E \longrightarrow \mathbb{R^+},$ the greedy algorithm produces a maximal member of $\mathcal{I}$ of maximum weight.
\end{thm}
\begin{proof}
Suppose $(E,\mathcal{I})$ is a matroid. Then $\emptyset \in \mathcal{I}$ and (I2) holds trivially.
And by \textit{theorem 5.2} we know that the greedy algorithm can find a maximal member $B \in \mathcal{B}$ of maximum weight if $(E,\mathcal{I})$ is a matroid.\\
Conversely, suppose $(E,\mathcal{I})$ is a pair satisfying $(I1),(I2)$ and $(G).$ Need to prove $\mathcal{I}$ satisfies $(I3)$ in order to have a matroid.\\
Suppose that (seeking a contradiction), if $I_1,I_2 \in \mathcal{I}$ with $|I_2|>|I_1|$ where there does not exist an $e \in I_2 \setminus I_1$ such that $I_1 \cup \{e\} \in \mathcal{I}.$\\
Now, $|I_1 \setminus I_2| < |I_2 \setminus I_1|$ and $I_1 \setminus I_2$ is non-empty.\\
So we can choose an $\epsilon>0$ such that
\begin{equation}
0 < (1+\epsilon)(|I_1 \setminus I_2|) < |I_2 \setminus I_1|
\end{equation}
Define $\omega:E \longrightarrow \mathbb{R^+}$ by:
\[
\omega(e) = \begin{cases}
               2 ,$ if $ e \in I_1 \cap I_2 \\
               \frac{1}{|I_1 \setminus I_2|} ,$ if $ e \in I_1 \setminus I_2\\
               \frac{1+\epsilon}{|I_2 \setminus I_1|},$ if $ e \in I_2 \setminus I_1 \\
               0, $ otherwise.$
            \end{cases}
\]\\
We need the greedy algorithm to fail for only one weight function to get our contradiction.
\begin{itemize}
\item The greedy algorithm will choose all the elements of $I_1 \cap I_2$ first as they are the heaviest elements.
\item Then it will choose all the elements of $I_1 \setminus I_2.$
\item By assumption, it cannot then pick any element of $I_2 \setminus I_1.$ Thus the remaining elements of $B_G$ will be in $E \setminus (I_1 \cup I_2).$\\
\end{itemize}
Hence,\[
\omega(B_G) = 2|I_1 \cap I_2| + |I_1 \setminus I_2| \bigg( \frac{1}{|I_1 \setminus I_2|} \bigg) \\
= 2|I_1 \cap I_2| + 1
\]\\
But by (I2), $I_2$ is contained in a maximal member $B_2$ of $\mathcal{I}$ and, $I_2 \subset B_2.$\\
\[
\omega(B_2) \geq \omega(I_2) = 2|I_1 \cap I_2| + |I_2 \setminus I_1| \bigg(\frac{1+\epsilon}{|I_2 \setminus I_1|} \bigg)\\
> 2|I_1 \cap I_2| + 1 = \omega(B_G)
\]\\
$\implies \omega(B_2) > \omega(B_G)$\\
Which means the greedy algorithm does not find a solution to our optimisation problem shown by \textit{theorem 5.2}, so the greedy algorithm fails for this weight function. We have a contradiction.\\
$\implies$ (I3) holds.\\
$\implies (E,\mathcal{I})$ is a matroid.
\end{proof}
\end{document}