\documentclass{article}
\usepackage[utf8]{inputenc}

\usepackage{amsmath,amsfonts,amssymb,amsthm,epsfig,epstopdf,titling,url,array, tikz,tkz-berge, calrsfs}
\usepackage{tkz-graph}

\title{Matroids And their Graphs}
\author{o.mcdonnell4@nuigalway.ie }
\date{February 2018}


\theoremstyle{plain}
\newtheorem{thm}{Theorem}[section]
\newtheorem{lem}[thm]{Lemma}
\newtheorem{prop}[thm]{Proposition}
\newtheorem*{cor}{Corollary}

\theoremstyle{definition}
\newtheorem{defn}{Definition}[section]
\newtheorem{conj}{Conjecture}[section]
\newtheorem{exmp}{Example}[section]

\theoremstyle{remark}
\newtheorem*{rem}{Remark}
\newtheorem*{note}{Note}

\newenvironment{rcases}
  {\left.\begin{aligned}}
  {\end{aligned}\right\rbrace}

\newcounter{excercise}
\newcounter{solution}
\newcounter{Question}

\newcommand\Excercise{%
  \textbf{Excercise:}~%
  \setcounter{solution}{0}%
}

\newcommand\TheSolution{%
  \textbf{Solution:}\\%
}

\newcommand\Question{%
    \textbf{Question:}~%
    \setcounter{Question}{0}%
}
\newcommand\Notation{%
  \textbf{Notation:}~%
}

\newcommand\Proof{%
    \textbf{Proof:}~%
}

\setlength{\droptitle}{-10em}

\begin{document}
\maketitle
 
 \section{Graph Theory Definitions}
 
\begin{defn}[Connected]
A graph is connected when there is a path between each pair of vertices.
\end{defn}
 
 
 \begin{defn}[Tree]
 A connected graph with no circuits.\\
 \noindent All subsets of E in $\mathcal{I}$ in M(G) the cycle matroid of G are the trees of the graph G.
 \end{defn}
 \begin{defn}[Forest]
 A disconnected graph with no circuits
 \end{defn}
 \begin{prop}
 A theorem of Cayley(1889) states that the number of distinct laballed trees which can be drawn using $n$ labelled points is $ n^(n-2)$.
 \end{prop}
 \begin{defn}[Rooted tree]
 A rooted tree has one vertex designated as an origin and called the root.
 \end{defn}
 \begin{defn}[Binary tree]
 A rooted tree in which the root has degree 2 and every other vertex has degree 1 or 3.
 \end{defn}
 
 \vspace{5mm}
 
\noindent The vertices are often labelled and referred to as nodes. Information may be recorded in them and a cost, penalty or probability may be associated with each edge. For example, the problem of joining all nodes in a graph by the minimum length of cable leads to a tree known as a \textit{minimum spanning tree}.
 
 \vspace{5mm}
 
\begin{defn}[Spanning Tree]
A spanning tree $T$ of an undirected graph $G$ is a subgraph that is a \textit{tree} which includes all of the vertices of $G$, with minimum possible number of edges.\\
\noindent A disconnected graph cannot contain a spanning tree as we cannot find a walk which brings us to all of the disconnected vertices.
\end{defn}
 
\begin{defn}[Acyclic]
An acyclic graph is a graph which contains no closed walks.
\end{defn}

\begin{defn}[Walk]
If there are vertices $v_{i-1}v_i$ for $i = 1, ..., n$ the sequence is called a walk.
\end{defn}

\begin{defn}[Closed Walk]
If $v_0 = v_n$ this sequence is called a closed walk.
\end{defn}

\begin{defn}[Bridge]
A bridge(or cut edge) is an edge of a graph whose deletion increases the numbe rof connected components. Equivalently, an edge is a bridge is and only if it is not contained in any cycle.
\end{defn}

\vspace{10mm}

\begin{lem}
Any acyclic graph on $n$ vertices has at most $n-1$ edges.
\end{lem}
\noindent\Proof Let G be an acyclic graph with n vertices.\\
\noindent if n = 1 then we have no edges hence nothing to prove.\\
\noindent Assumse $n>1,$ let $e$ be an edge in $G$ connecting two vertices $ab$ in the vertex set of $G.$\\
\noindent Let $H = G \setminus \{e\}, H$ has one more connected component than $G.$ $H$ has two maximal acyclic connected componenets, and thus can be decomposedinto acyclic connected graphs $H_1, H_2, ..., H_k where k \geq 2.$\\
\noindent By induction, we can assume each graph $H_i$ contains at most $n_i - 1$ edges where $n_i$ is the number of vertices of $H_i.$\\
\noindent Then $G$ has at most $n-1$ edges.\\
\noindent $(n_1 - 1) + ... + (n_k - 1) + 1 = (n_1 + ... + n_k) - (k - 1) \leq n - 1$ edges.
\\ \qed

 \begin{lem}
 Let $G$ be a graph. Then the following conditions are equivalent:\\
 1) $G$ is a tree.\\
 2) $G$ does not contain any cycles, but adding any further edge yields a cycle.\\
 3) Any two vertices of $G$ are connected by a unique path.\\
 4) $G$ is connected, and any edge of $G$ is a bridge.
 \end{lem}
 
 \noindent\Proof $(1) \Longrightarrow (2)$\\
 \noindent Suppose that $G$ is a tree,then $G$ is a connected graph with no circuits. Let $e$ be a new edge in G with $e = g_ig_k$ where $g_i,g_k$ are in the vertex set of $G.$ Then as $G \cup \{e\}$ must be connected, there exists a walk between any pair of vertices of $G$. So there is a walk $K$ from $g_j \rightarrow g_k$ and there is also a walk $L$ from $g_k \rightarrow g_j$ where $K$ does not traverse $e$ and $L$ does traverse $e$ and so we have a cycle.\\

\noindent\Proof $(2) \Longrightarrow (3)$\\
\noindent Let $u,v$ be vertices of $G.$ If there was not path joining $uv$ in $G$ then $e = uv$ does not create a cycle in $G.$ Thus $G$ must be connected.\\
\noindent Suppose $G$ contained two different paths $W_1, W_2$ from $u$ to $v.$\\
\noindent Then $ u \longrightarrow v \longrightarrow u$ would be a closed walk in $G. \\ \implies G$ contains a cycle. Which is a contradiction.\\

\noindent\Proof $(3) \Longleftrightarrow (4)$\\
\noindent $G$ is connected by hypothesis. Let $e = uv$ be an edge in $G.$\\
\noindent Suppose $e$ is not a bridge, then $G \setminus \{e\}$ is still connected. But then we have two distinct path from u to v in G.\\

\noindent\Proof $(4) \Longrightarrow (1)$\\
\noindent G is connected by hypotheis.\\
\noindent Suppose $G$ contains a cycle $K.$ Then any edge of $K$ could be ommited from $G,$ and the resulting graph would still be connected. In other words, no edge of $K$ would be a bridge, a contradiction.\\
\qed

\begin{defn}
let $(G,\omega)$ be a network. For any subset $T$ of the edge set of $G,$\\
\indent\indent\indent\indent $\omega (T) = \displaystyle\sum_{e \in T} \omega (T)$\\
\noindent is called the weight of $T.$
\end{defn}

\begin{defn}[Minimal Spanning Tree]
A spanning tree is a \textit{minimal} spanning tree if its weight is minimal of all the weights of spanning trees.

\noindent A forest can be considered by finding a minimal spanning tree for each connected component of $G.$
\end{defn}

\begin{rem}
If the weight $\omega$ is constant, any spanning tree is minimal.\\
\noindent In this case, determining a minimal spanning tree could be done using a breadth-first search.
\end{rem}

\begin{rem}
Determining the number of spanning trees of a graph in polynomial time is NP-hard.
\end{rem}

\begin{note}
Let $T$ be a tree and $e$ an edge not in $T.$ According to lemma 1.3 the graph arising from $T \cup \{e\}$ contains a unique cycle. We denote this cycle by $C_T(e)$
\end{note}

\begin{thm}[Without Proof]
Let $(G,\omega)$ be a network, where $G$ is a connected graph. A spanning tree $T$ of $G$ is minimal if and only if, for each edge $e$ in $G \setminus T$, we have,\\
\indent\indent\indent $\omega(e) \geq \omega(f) \forall$ edges $f$ in $C_T(e).$
\end{thm}

\vspace{3mm}

Another characterisation of minimal spanning trees exist. But the following defintions are required first.

\begin{defn}[Cut]
Let $G$ be a graph having vertex set $V.$ A \textit{cut} is a partition $S = X \dot{\cup} X'$ of $V$ into two non-empty subsets.
\end{defn}

\begin{defn}[Cocycle]
By $E(s)$ or $E(X,X')$ we denote the set of all edges incident with one vertex in $X$ and one vertex in $X'.$ Any such edge set is called a cocycle.
\begin{rem}
We are concerned with cocycles that are constructed from trees.
\end{rem}
\end{defn}


 \end{document}