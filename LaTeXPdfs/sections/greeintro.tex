\documentclass[../main.tex]{subfiles}
\begin{document}
\noindent Matroids have now been shown to be a hereditary set-system which naturally works well with the greedy algorithm, guaranteeing  optimal solution to optimisation problems. We have also seen in \textit{theorem 5.6} that matroids are the only hereditary set-system that have this property. But what about non-hereditary set-systems? Are their other systems that yield optimal solutions to optimisation problems by way of the greedy algorithm? and if so how much structure must be imposed on the set-systems to guarantee this property?
These are the questions we try to explore using Jungnickel's text.\cite{jungnickel} In this section, proofs are omitted as they are also omitted in the text and are beyond the scope of this project. For proofs or further investigation, refer to \cite{bryant_brooksbank}, \cite{kortez__lovasz}, \cite{greedoids}.
\end{document}