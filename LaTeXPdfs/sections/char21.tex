\documentclass[../main.tex]{subfiles}
\begin{document}
\noindent And now similar to with circuits, we can characterise a matroid by way of it's bases as proved below. 
\begin{defn}
A base is a maximally independent subset of $\mathcal{I}.$\\
\noindent As seen previously all maximally independent sets in a matroid have the same cardinality.
\end{defn}
\begin{defn}
The rank of a matroid is equal to the cardinality of it's bases. The rank is denoted as $r(M)$ for a matroid M.
\end{defn}
\begin{thm}
 Let $\mathcal{B}$ be a set of subsets of a finite set E. Then $\mathcal{B}$ is the collection of bases of a matroid on E if and only if $\mathcal{B}$ satisfies the following conditions:\\
 (B1) $\mathcal{B}$ is non-empty.\\
 (B2) If $B_1$ and $B_2$ are members of $\mathcal{B}$ and $x \in B_1 \setminus B_2$, then there is an element $y$ of $B_1 \setminus B_2$ such that $(B_1 \setminus \{x\}) \cup \{y\} \in \mathcal{B}$.
 \end{thm}

\begin{proof}
\noindent By (I1) $\emptyset$ is always independent, so $\mathcal{B}$ must always contain at least the $\emptyset$, (B1) holds.\\
\noindent Let $B_1 ,B_2 \in \mathcal{B}, B_1 \neq B_2.$ \\
\noindent $|B_1| = |B_2|$ so (I3) does not directly apply here, as any element of cardinality larger that $|B_i|,$ where $B_i \in \mathcal{B}$ will be contained in $\mathcal{C}$(the set of circuits).\\
\noindent Let $x \in B_1 \setminus B_2 \implies x \in B_1, x \notin B_2.$\\
\noindent $|B_1| = |B_1 \setminus \{x\}| + 1 \implies B_1 \setminus \{x\} \in \mathcal{I}$ but not in $\mathcal{B}.$\\
\noindent $|B_2| = |B_1 \setminus \{x\}| + 1$ so now we can apply (I3).\\
\noindent Therefore, there exists a $y \in B_2 \setminus B_1$ such that $(B_1 \setminus \{x\}) \cup \{y\} \in \mathcal{I}$\\
$|(B_1 \setminus \{x\}) \cup \{y\}| = |B_1 \setminus \{x\}| + 1 = |B_1| = ... = |B_r|$ where the $B_i$ are maximal elements of $\mathcal{I}$ and have the same cardinality.\\ $\implies (B_1 \setminus \{x\}) \cup \{y\}$ is maximal in $\mathcal{I}.$\\
$\implies (B_1 \setminus \{x\}) \cup \{y\} \in \mathcal{B}$, (B2) holds.

\vspace{4mm}

\noindent Conversely, suppose that $\mathcal{B}$ satisfies (B1) and (B2) (this direction is much more difficult and appears in Oxley's text\cite{ox_book}).\\
\noindent By (B1), $\mathcal{B}$ is always non-empty which shows (I1) holds.
\\
\noindent By definition, a base $B_1 \in \mathcal{B}$ is a maximally independent subset of E. Then for all $B_i \in \mathcal{B}$ the subsets $b_{i,k} \subseteq B_i$  are independent. Therefore all the $b_{i,k}$ are in $\mathcal{I}.$\\ \noindent $\implies$ (I2) holds. Showing a matroid can be generated through the bases.
\\
\noindent Next, assume that (I3) fails. That, for $I_1, I_2 \in \mathcal{I}$, where $|I_1|<|I_2|,$\\ \noindent there does not exist a  $y \in I_1 \setminus I_2$ such that $I_2 \cup \{y\} \in \mathcal{I}.$\\
\noindent Let $B_1, B_2 \in \mathcal{B}, |B_1| = |B_2|$ such that $I_1 \subseteq B_1$ and $I_2 \subseteq B_2.$\\
Assume $B_2$ is chosen in such a way as to ensure that $|B_2 \setminus (I_2 \cup B_1)|$ is minimal. By above choice of $I_1,I_2$ we have:\\
\begin{equation}
I_2 \setminus B_1 = I_2 \setminus I_1
\end{equation}

Now suppose that $B_2 \setminus (I_2 \cup B_1)$ is non-empty. Then we can choose an element $x$ from this resulting set and apply $(B2)$, which says there is an element $y$ in $B_1 \setminus B_2$ such that $(B_2 \setminus \{x\}) \cup \{y\} \in \mathcal{B}.$ But then\\
\begin{equation}
|((B_2 \setminus \{x\}) \cup \{y\}) \setminus (I_2 \cup B_1)| < |B_2 \setminus (I_2 \cup B_1)|
\end{equation}
\noindent And thus the choice of $B_2$ is contradicted. Hence $B_2 \setminus (I_2 \cup B_1)$ is empty and so $B_2 \setminus B_1 = I_2 \setminus B_1.$ Therefore,
\begin{equation}
B_2 \setminus B_1 = I_2 \setminus I_1
\end{equation}
\noindent Next we must show that $B_1 \setminus (I_1 \cup B_2)$ is empty. If not then $(B2)$ can be applied and $(B_1 \setminus \{x\}) \cup \{y\} \in \mathcal{B}.$\\
Now, $I_1 \cup \{y\} \subseteq  (B_1 \setminus \{x\}) \cup \{y\}$ so $I_1 \cup \{y\} \in \mathcal{I}.$ Since $y \in B_2 \setminus B_1$, it follows by (3) that $y \in I_2 \setminus I_1$ and so we reach a contradiction to our inital assumption, We conclude that $B_1 \setminus (I_1 \cup B_2)$ is empty. Hence $B_1 \setminus B_2 = I_1 \setminus B_2.$\\
Since the last set is contained in $ I_2 \setminus I_1$, it follows that
\begin{equation}
I_2 \setminus B_1 \subseteq I_2 \setminus I_1
\end{equation}
As all the bases have the same cardinality so by (3) and (4) we see that\\ $I_1 \setminus I_2 \geq I_2 \setminus I_1$ so $|I_1| \geq |I_2|.$ This contradicts our choice of independent sets. And so we have that the pair $(E,\mathcal{I})$ is a matroid.
\end{proof}
\end{document}