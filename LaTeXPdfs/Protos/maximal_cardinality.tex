\documentclass{article}
\usepackage[utf8]{inputenc}

\usepackage{amsmath,amsfonts,amssymb,amsthm,epsfig,epstopdf,titling,url,array, tikz,tkz-berge, calrsfs}
\usepackage{tkz-graph}

\title{Matroids And their Graphs}
\author{o.mcdonnell4@nuigalway.ie }
\date{19 January 2018}


\theoremstyle{plain}
\newtheorem{thm}{Theorem}[section]
\newtheorem{lem}[thm]{Lemma}
\newtheorem{prop}[thm]{Proposition}
\newtheorem*{cor}{Corollary}

\theoremstyle{definition}
\newtheorem{defn}{Definition}[section]
\newtheorem{conj}{Conjecture}[section]
\newtheorem{exmp}{Example}[section]

\theoremstyle{remark}
\newtheorem*{rem}{Remark}
\newtheorem*{note}{Note}

\newenvironment{rcases}
  {\left.\begin{aligned}}
  {\end{aligned}\right\rbrace}

\newcounter{excercise}
\newcounter{solution}
\newcounter{Question}

\newcommand\Excercise{%
  \textbf{Excercise:}~%
  \setcounter{solution}{0}%
}

\newcommand\TheSolution{%
  \textbf{Solution:}\\%
}

\newcommand\Question{%
    \textbf{Question:}~%
    \setcounter{Question}{0}%
}
\newcommand\Notation{%
  \textbf{Notation:}~%
}

\newcommand\Proof{%
    \textbf{Proof:}~%
}

\setlength{\droptitle}{-10em}

\begin{document}
\maketitle
 
 \section{Cardinality of maximal independent sets}
 
 \begin{thm}
 Show that if $\mathcal{I}$ is a non-empty hereditary set of subsets of a finite set E, then $(E,\mathcal{I})$ is a matroid if and only if, for all $X \subset E$, all maximal members of $\{I : I \in \mathcal{I} $ and $ I \subset X\}$ have the same number of elements.
 \end{thm}
 
\noindent \Proof $(\implies)$ Let $B_1 , B_2$ be maximal elements of $\{I : I \in \mathcal{I} $and $ I \subset X\}$ \\
\noindent And assume $|B_1| < |B_2|$
Then since $B_1, B_2 \in \mathcal{I}$
\\
There exists $e \in (B_2 \setminus B_1)$ such that $B_1 \cup \{e\} \in \mathcal{I}$
\\
This contradicts our maximality of $B_1$.\\

 \noindent $\implies$ All maximal elements of the set $\{I : I \in \mathcal{I} $ and $ I \subset X\}$ in our matroid M have the same cardinality.
 \\ \qed
 \end{document}